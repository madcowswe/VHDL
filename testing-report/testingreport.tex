% !TEX TS-program = pdflatex
% !TEX encoding = UTF-8 Unicode

% This is a simple template for a LaTeX document using the "article" class.
% See "book", "report", "letter" for other types of document.

\documentclass[]{article}

\usepackage[utf8]{inputenc} % set input encoding (not needed with XeLaTeX)

%%% PAGE DIMENSIONS
\usepackage{geometry} % to change the page dimensions
\geometry{a4paper} % or letterpaper (US) or a5paper or....
% \geometry{margin=1in} % for example, change the margins to 2 inches all round
% \geometry{landscape} % set up the page for landscape
%   read geometry.pdf for detailed page layout information

\usepackage{graphicx} % support the \includegraphics command and options

% \usepackage[parfill]{parskip} % Activate to begin paragraphs with an empty line rather than an indent

%%% PACKAGES
\usepackage{booktabs} % for much better looking tables
\usepackage{array} % for better arrays (eg matrices) in maths
\usepackage{paralist} % very flexible & customisable lists (eg. enumerate/itemize, etc.)
\usepackage{verbatim} % adds environment for commenting out blocks of text & for better verbatim
\usepackage{subfig} % make it possible to include more than one captioned figure/table in a single float
\usepackage{microtype} %makes awesome kerning and punctuation come half way out the edge of the text
\usepackage{listings} %for code listings
\usepackage{color} %for colored syntax highligting

%%% Code listing
\definecolor{mygreen}{rgb}{0,0.6,0}
\definecolor{mygray}{rgb}{0.5,0.5,0.5}
\definecolor{mymauve}{rgb}{0.58,0,0.82}
\lstset{breaklines=true,
basicstyle=\footnotesize\ttfamily,
commentstyle=\color{mygreen},
keywordstyle=\color{blue},
numberstyle=\tiny\color{mygray},
tabsize=2,
language=c
}

% to include a file as a listing: \lstinputlisting{intio.c}
% inline listing: \begin{lstlisting}[frame=single]

%%% HEADERS & FOOTERS
\usepackage{fancyhdr} % This should be set AFTER setting up the page geometry
\pagestyle{fancy} % options: empty , plain , fancy
\renewcommand{\headrulewidth}{0pt} % customise the layout...
\lhead{}\chead{}\rhead{}
\lfoot{}\cfoot{\thepage}\rfoot{}

%%% ToC (table of contents) APPEARANCE
\usepackage[nottoc,notlof,notlot]{tocbibind} % Put the bibliography in the ToC
\usepackage[titles,subfigure]{tocloft} % Alter the style of the Table of Contents
%\renewcommand{\cftsecfont}{\rmfamily\mdseries\upshape}
%\renewcommand{\cftsecpagefont}{\rmfamily\mdseries\upshape} % No bold!
\usepackage{hyperref} % use hyperlinked ToC
\hypersetup{colorlinks=true, linkcolor=black}

%%%-------------------------------------------------------------------


\title{Title Here}
\author{Oskar Weigl - ow610\\ and \\ Ryan Savitski - rs510}
%\date{} % Activate to display a given date or no date (if empty),
         % otherwise the current date is printed 

\begin{document}
\maketitle

%\renewcommand{\abstractname}{Summary}
%\begin{abstract}
%	Write the abstract here
%\end{abstract}

\tableofcontents
\clearpage

\section{Operation}
The following features are implemented:
\begin{itemize}
	\item Line drawing:
	\begin{itemize}
		\item All colors
		\item All directions
		\item Any length (including one pixel and entire screen)
	\end{itemize}
	\item Fills (Clearscreen)
	\begin{itemize}
		\item All colors
		\item All directions
		\item Any size
		\item Crossing any cache word boundary
	\end{itemize}
\end{itemize}

\section{Tests} % (fold)
\label{sec:tests}

\subsection{First Test} % (fold)
\label{sub:first_test}

This inital test tests the basic operation of the hardware, with no special corner case consideration. It tests the following:
\begin{itemize}
	\item Drawing lines in all 8 octants (including horizontal/vertical).
	\item White/black/invert lines.
	\item Basic fill with black color with coordinates specified as bottom left to top right.
\end{itemize}

In terms of hardware, this tests:
\begin{itemize}
	\item That commands are handled properly.
	\item That interfaces are not violated.
	\item That the line drawing fsm and caching fsm are working.
	\item That color selection works for lines.
	\item That basic filling works.
\end{itemize}

% subsection first_test (end)

\subsection{Second Test} % (fold)
\label{sub:second_test}

second test:
multipoint lines (starting line segment from previous line endpoint)
black/white/invert fills

=====
for hw this tests.....
in addition to the above, tested content:
 state of cursor saved and handled properly in db
 color selection for fills
==============================

% subsection second_test (end)

% section tests (end)

\section{Ryans Unifinished Stuff} % (fold)
\label{sec:ryans_unifinished_stuff}


third test:
fills in all directions (i.e. the coordinates are no longer always bottom left to top right)
================
tests in hw....
 fill logic in rcb can handle all directions of start and end coordinate points
================================
fourth test:
single pixel lines/fills
single pixel wide fills
===============
tests:
 special handling of single pixel lines/fills in the db
 handling of 1-pixel wide fills in rcb
 
=================================
fifth test: 
long lines

=======
tests:
 whether draw entity has overflows in errors for long lines (note: atm doesn't check it extensively)
=======


????
fills inside one cache
fills that cross 4 caches
multiple inversions

% section ryans_unifinished_stuff (end)

\end{document}
